\documentclass{webofc}
\usepackage[varg]{txfonts}   % Web of Conferences font




\begin{document}
%
\title{A Python upgrade to the GooFit package for parallel fitting}

\author{
	    \firstname{Henry}
        \lastname{Schreiner}\inst{1}\fnsep\thanks{\email{henry.fredrick.schreiner@cern.ch}}
        \and
        \firstname{Himadri}
        \lastname{Pandey}\inst{1}
        \and 
        \firstname{Micheal D}
        \lastname{Sokoloff}\inst{1}
        \and
        \firstname{Hittle}
        \lastname{Bradley}\inst{2}
        \and 
        \firstname{Karen}
        \lastname{Tomko}\inst{2}
        \and 
        \firstname{Christoph}
        \lastname{Hasse}\inst{3}
}        % etc.

\institute{
	University of Cincinnati
\and
	Ohio Supercomputer Center
\and
	CERN / Technische Universit\"at Dortmund (DE)
}

\abstract{%
  The GooFit highly parallel fitting package for GPUs and CPUs has been substantially upgraded in the past year. Python bindings have been added to allow simple access to the fitting configuration, setup, and execution. A Python tool to write custom GooFit code given a (compact and elegant) MINT3/AmpGen amplitude description allows the corresponding C++ code to be written quickly and correctly. New PDFs have been added. The most recent release was built on top of the December 2017 2.0 release that added easier builds, new platforms, and a more robust and efficient underlying function evaluation engine.
}
%
\maketitle
%
\section{Introduction}
\label{intro}
High Energy Physics experiments around the world are producing record amounts of data. Existing tools, such as the RooFit fitting framework, provide flexible and powerful abstractions for building distributions to fit that data with, but this power comes at a cost; this is computationally expensive, and often only runs on a single core. Modern architectures provide many cores, as well as new computing paradigms, such as GPUs, that provide significant new potential for high performance computations, but are non-trivial for the physicst to use to build distributions in the description style familiar to him.

GooFit is a high-performance multi-thread and GPU ready framework providing a similar syntax to RooFit developed at the University of Cincinnati in 2013. (TODO: REF) It provided composition of model pieces in the same manor as RooFit, but was powered by GPUs. It also provided ready to use models The GooFit 2.0 release added a simpler build process, making it easy for users to pick up and run GooFit code.

\section{Python Bindings}
\label{sec-py}
More here.

\section{The New Indexing System}
\label{sec-ind}
More here.

\section{A Uniform Decay Language}
\label{sec-ampgen}
More here.


\section{Summary}
\label{sec-summary}
More here.

%For one-column wide figures use syntax of figure~\ref{fig-1}
%\begin{figure}[h]
% Use the relevant command for your figure-insertion program
% to insert the figure file.
%\centering
%\includegraphics[width=1cm,clip]{tiger}
%\caption{Please write your figure caption here}
%\label{fig-1}       % Give a unique label
%\end{figure}

%For two-column wide figures use syntax of figure~\ref{fig-2}
%\begin{figure*}
%\centering
%% Use the relevant command for your figure-insertion program
%% to insert the figure file. See example above.
%% If not, use
%\vspace*{5cm}       % Give the correct figure height in cm
%\caption{Please write your figure caption here}
%\label{fig-2}       % Give a unique label
%\end{figure*}
%
%For figure with sidecaption legend use syntax of figure
%\begin{figure}
%% Use the relevant command for your figure-insertion program
%% to insert the figure file.
%\centering
%\sidecaption
%\includegraphics[width=5cm,clip]{tiger}
%\caption{Please write your figure caption here}
%\label{fig-3}       % Give a unique label
%\end{figure}

%For tables use syntax in table~\ref{tab-1}.
%\begin{table}
%\centering
%\caption{Please write your table caption here}
%\label{tab-1}       % Give a unique label
%% For LaTeX tables you can use
%\begin{tabular}{lll}
%\hline
%first & second & third  \\\hline
%number & number & number \\
%number & number & number \\\hline
%\end{tabular}
%% Or use
%\vspace*{5cm}  % with the correct table height
%\end{table}
%
% BibTeX or Biber users please use (the style is already called in the class, ensure that the "woc.bst" style is in your local directory)
% \bibliography{name or your bibliography database}
%
% Non-BibTeX users please use
%
\begin{thebibliography}{}
%
% and use \bibitem to create references.
%
\bibitem{RefJ}
% Format for Journal Reference
Journal Author, Journal \textbf{Volume}, page numbers (year)
% Format for books
\bibitem{RefB}
Book Author, \textit{Book title} (Publisher, place, year) page numbers
% etc
\end{thebibliography}

\end{document}

% end of file template.tex

<div id='footer'><table width='100%'><tr><td class='right'><a href='http://fusioninventory.org/'><span class='copyright'>FusionInventory 9.1+1.0 | copyleft <img src='/glpi/plugins/fusioninventory/pics/copyleft.png'/>  2010-2016 by FusionInventory Team</span></a></td></tr></table></div>
